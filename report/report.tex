\documentclass[a4paper,11pt]{article}
\usepackage{fullpage}
\usepackage[colorlinks=true]{hyperref}
\usepackage{amssymb}
\usepackage{amsmath}
\usepackage{fancyhdr}
\usepackage{verbatim}
\usepackage{lmodern}
\usepackage{listings}
\usepackage{graphicx}
\usepackage{booktabs}
\usepackage{caption}
\usepackage{color}

\setlength{\headheight}{26pt}
\setlength{\headsep}{0.2in}

\definecolor{gray}{rgb}{0.5, 0.5, 0.5}

\lstset{
    language=Python,
    basicstyle=\footnotesize,
    numbers=left,
    numberstyle=\scriptsize\color{gray},
    stepnumber=1,
    numbersep=5pt,
    backgroundcolor=\color{white},
    showspaces=false,
    showstringspaces=false,
    rulecolor=\black,
    tabsize=4,
    captionpos=t,
    breaklines=true,
    keywordstyle=\color{black},
    commentstyle=\color{black},
    stringstyle=\color{black}
}

\title{OpenCUDA+MPI\\
       A Framework for Heterogeneous GP-GPU Distribute Computing}
\date{\today}
\author{Kenny Ballou\\
        College of Engineering -- Department of Computer Science\\
        Boise State University\\
        Alark Joshi, Ph. D}

\lhead{\fancyplain{}{Student Research Initiative 2013\\
                     Progress Report}}
\pagestyle{empty}

\begin{document}
\begin{titlepage}
\begin{center}

\textsc{\LARGE OpenCUDA+MPI}\\[1.5cm]
\textsc{\Large A Framework for Heterogeneous
               GP-GPU Distributed Computing}\\[1.5cm]

Kenny Ballou\\
Nilab Mohammad Mousa\\
College of Engineering -- Department of Computer Science\\
Boise State University\\
Alark Joshi, Ph.D\\
\vfill
\end{center}
\end{titlepage}

\nocite{*}
\thispagestyle{fancy}
\abstract{The introduction and rise of General Purpose Graphics Computing has
significantly impacted parallel and high-performance computing. Even worse, it
has brought about even more challenges when it comes to distributed computing
(with GPU's). Current solutions target specifics: specific hardware, specific
network topology, a specific level of sameness. What if we could ignore
specifics? What if we could write a general algorithm to solve a problem and
have a job scheduler do the rest? That is the goal of OpenCUDA+MPI. To achieve
this goal, we have written a framework that allows a developer/ data scientist
to write a general algorithm/ solution without the overhead of worrying about
the specifics of the cluster it will run against. We have found, [nothing yet]
and, therefore, can conclude [also nothing].}\\

\textbf{Keywords:} Parallel Computing, Distributed Computing, General Purpose
Graphics Processing, High-Performance Computing, Scientific Computing
\section{Introduction}

\Glspl{gpu} have significantly altered the way high-performance computing tasks
can be performed today. A major challenge remains in being able to seamlessly
intergrate multiple workstations to further parallelize the computational
tasks. Current approaches provide the ability to parallelize tasks, but they
are less focused on optimally utillizing the varied capabilities of
heterogeneous graphics cards in the cluster of workstations.\\

\subsection{Outline}

\subsection{What Others Have Done}

% vim: syntax=tex:
\section{Methodology}

Our framework will be developed using a number of tools and will also build
upon a few already developed and mature software libraries. Overall, we plan to
use the \texttt{Python} programming language, \gls{mpi} for cross process
communication, and \gls{cuda} for graphics computing.

\subsection{Python}

We will use \texttt{Python} programming language because of its ease of
development and plethora of existing libraries such as \texttt{mpi4py} and
\texttt{pyCUDA}. \texttt{Python} also adds some advantages when it comes to
usability when needing more performance. For example, if we discover a need
for certain aspects of the project to be tuned or otherwise run faster, we can
easily switch to C or C++ and write sections of the program in a (more
performant) native language.

\subsection{MPI}

Currently our cluster consists of 16 computers provided to our research lab by
the Computer Science department of Boise State University. We will use
\gls{mpi} because it has established itself as the de-facto interface for cross
process communication \cite{website:MPI-Tutorial}. To allow for interfacing
with \texttt{Python} \cite{website:mpi-4-python}, we will use \texttt{mpi4py}
because of the library's maturity and implementation completeness.

\subsection{CUDA}

We will be using \gls{cuda}, and in particular, \texttt{pyCUDA}, because of
existing knowledge of the library/ framework and also because it is a well
established library for \gls{gpu} computing.

\subsection{\texttt{OpenCUDA+MPI}}

Our frameworks' goal is to make accessible the power of \gls{cluster} computing
without necessarily knowing in-depth the complexities of inter-computer related
communication. Further, in doing so, we would like to expose functionality that
may not be available even in more established \gls{cluster} libraries and
frameworks. Namely, facilities for debugging and profiling.

\subsection{Testing}

To evaluate the efficacy and accuracy of our framework several tests will be
executed. As previously mentioned one of the first algorithms to be tested will
be the problem of vessel extraction. Accurately extracting vascular structures
from a Computerized Tomographic angiography--also called CT angiography
scans--is important for creating oncologic surgery planning tools as well as
medical visualization applications \cite{erdt2008automatic}.  Currently, we use
a single \gls{gpu} to extract vascular structures from a CT angiography scan,
which is computationally intensive.

The following test programs will be developed as time allows:

\begin{itemize}
    \item N-Body Simulation
    \item Prime Number Searching
\end{itemize}

Every test program will be evaluated in three categories: \gls{cpu}-only,
\gls{cuda}-only, \gls{cuda} and \gls{mpi} and \texttt{OpenCUDA+MPI}. Having all
of the prior solutions that do \emph{not} use the framework provides a baseline
time comparison and provide immense insights into the pains and difficulties we
are actually attempting to solve.

%% vim: syntax=tex:
\section{Results}

\subsection{Vector Summation}

Our first developed test program was a $ 10 $ billion element wise vector
summation problem. This is a simple and, as we will see, a bad example of using
a cluster to speed up the computational time required. Although we did see a
increase in performance, the cost of \gls{io} far out weighs the benefit.
Specifically, to do the computation on one machine (one \gls{cpu}), it took a
\gls{wall_time} of $254$ seconds (about $4$ minutes) and a \gls{cpu_time} of
$13.7$ seconds.  Further, to do the computation using a single \gls{gpu} took
about $4172$ seconds (\gls{wall_time}) or about $70$ minutes, $13.83$ seconds
(\gls{cpu_time}) while the computing the summation of the cluster took about
$3177$ seconds (\gls{wall_time}) or about $50$ minutes, $10.51$ seconds
(\gls{cpu_time}). Our \gls{cpu} only implementation took the least amount of
elapsed time, it took the second longest \gls{cpu_time}. Our \gls{cuda} only
implementation took the longest in both \gls{wall_time} and \gls{cpu_time} and
our cluster implementation was shortest \gls{cpu_time} but seconds longest
elapsed time. The benefit of saving an upwards of $3.3$ seconds is not worth
the extra incurred cost of $2923$ seconds. Not so surprisingly though, running
the vector summation problem over the cluster \emph{without} using \gls{cuda}
does increase the overall elapsed time of the problem; namely, it took $226$
seconds \gls{wall_time} (currently the correct \gls{cpu_time} is unable to be
collected). Further, increasing the number of nodes part of the program's pool,
decreases the \gls{wall_time} for each node.

\begin{table}[htb]
\centering{}
\begin{tabular}{lcc}
\toprule{}
\textbf{Method} & \textbf{Time (s)} & \textbf{Total Time (s)} \\
\midrule{}
CPU Only & 13.7 & 254.13 \\
\midrule{}
CUDA (Single \Gls{node}) & 13.83 & 4172 \\
\midrule{}
MPI + CUDA (7 \glspl{node}) & 10.51 & (average) 3177 \\
\midrule{}
MPI (7 \glspl{node}) & & (average) 226  \\
\midrule{}
MPI (16 \glspl{slot}) & & (average) 149 \\
\bottomrule{}
\end{tabular}
\caption{Computational Timing Comparison of $ 10^9 $ element wise vector
summation}
\end{table}

\subsection{N-Body Problem}

We have several sizes of the \texttt{N-Body} problem that we tested with:
$2,001$ particles, $20,000$ particles, $200,000$ particles, $2,000,000$, and
$20,000,000$ particles.\\

The computational times are for only one time step. The method for computing
the gravitational potential is an adaptation of the \gls{p3m} method. The
benefit of using this method is we are able to nicely distribute the problem
over the \gls{cluster} and / or over a \gls{gpu} (because of memory
limitations) while maintaining a respectable accuracy for close body
interactions. Further, if a grid contains more bodies than a specified
threshold (in our case $200,000$), we can further sub-divide the grid to
improve performance and maintain accuracy still.\\

There are other algorithms for computing \texttt{N-Body} problems on the
\gls{cpu} that are quite efficient, notably, the Barnes-Hut Tree
algorithm\cite{barneshut1986}; however, using it would distort and confound the
comparisons between \gls{cpu}, \gls{gpu}, and \gls{cluster} implementations,
not to mention the complexities of implementing such an algorithm on
\glspl{gpu} and over a \gls{cluster}.

\subsubsection{N-Body --- CPU}

In \gls{cpu} tests, we were only able to complete several of the problem sizes;
the larger sizes are infeasible. Notably, the smaller sizes were computed in a
relatively respectable amount of time. While the bigger sizes were time
consuming, not even attempted, or aborted. For example, the 2 million body
problem was aborted after running for about 2 weeks.

\begin{table}[htb]
\centering{}
\begin{tabular}{lccc}
\toprule{}
\textbf{Size} & \textbf{User (seconds)} &
\textbf{Sys (seconds)} & \textbf{Real (seconds)} \\
\midrule{}
2001          & 28.81   & 0.02    & 30.77   \\
\midrule{}
20000         & 2382.40 & 2.27    & 2393    \\
\midrule{}
200000        & 113983  & 34.45   & 114349  \\
\midrule{}
2 million     & aborted & aborted & aborted \\
\midrule{}
20 million    & N/A     & N/A     & N/A     \\
\bottomrule{}
\end{tabular}
\caption{\gls{cpu} N-Body simulation using particle-particle method}
\label{tab:cpu_nbody}
\end{table}

\subsubsection{N-Body --- GPU}

As noted above, the \gls{gpu} (\gls{cuda}) implementation uses the same
computational method (\gls{p3m}). Using the \gls{gpu}, we were able to compute
the $ 20,000,000 $ size problem and may be able to compute larger sets within a
\emph{reasonable} amount of time. See \cref{tab:gpu_nbody} for a breakdown of
the running times.

\begin{table}[htb]
\centering{}
\begin{tabular}{lccc}
\toprule{}
\textbf{Size} & \textbf{User (seconds)} &
\textbf{Sys (seconds)} & \textbf{Real (seconds)} \\
\midrule{}
2001          & 10.08   & 1.52    & 14.14  \\
\midrule{}
20000         & 22.30   & 2.46    & 26.82  \\
\midrule{}
200000        & 44.63   & 4.84    & 52.86  \\
\midrule{}
2 million     & 186.59  & 21.23   & 217.60 \\
\midrule{}
20 million    & 1289.24 & 159.29  & 1510   \\
\bottomrule{}
\end{tabular}
\caption{Single \gls{gpu} (\gls{cuda}) N-Body simulation using \gls{p3m}
method}
\label{tab:gpu_nbody}
\end{table}

\subsubsection{N-Body --- 16 Node Cluster}

Similar to the other solutions, we are still using the \gls{p3m} method for
computing a single time step of the N-Body problem. Over 16 nodes, we were able
to see drastic improvements over \gls{cpu} and \gls{cuda} implementations.
Notably, in the $20,000$ problem size, the \gls{cluster} did nearly $19044\%$
better than the \gls{cpu} and about $115\%$ better than a single \gls{gpu}.
Further, in the $200,000$ problem size, we see the \gls{cluster} did about
$535,743\%$ better versus \gls{cpu} and about $148\%$ better versus \gls{gpu}.
See \cref{tab:cudampi_nbody} for times of each problem size.

\begin{table}[htb]
\centering{}
\begin{tabular}{lccc}
\toprule{}
\textbf{Size} & \textbf{User (seconds)} &
\textbf{Sys (seconds)} & \textbf{Real (seconds)} \\
\midrule{}
2001          & N/A     & N/A      & N/A     \\
\midrule{}
20000         & 0.17    & 0.032    & 12.50   \\
\midrule{}
200000        & 0.147   & 0.028    & 21.34   \\
\midrule{}
2 million     & 0.15    & 0.025    & 97.76   \\
\midrule{}
20 million    & 0.15    & 0.06     & 950.045 \\
\bottomrule{}
\end{tabular}
\caption{16 node \gls{cluster} N-Body simulation using \gls{p3m} method}
\label{tab:cudampi_nbody}
\end{table}

%\include{Discussion}
%% vim: syntax=tex:
\section{Conclusions}

We have developed some baseline test solutions to a few problems. From the
baseline solutions, we can easily tell there are significant performance
increases from parallelizing code. However, there is still a complexity cost.
The goal of \texttt{OpenCUDA+MPI} is to limit this complexity cost and from our
really early versions of the framework, the outlook of being able to do just
that looks very good.

\subsection{Future Work}

Work is continuing to progress on the development of the framework, but an
early alpha version is nearly complete. As we continue to develop the
framework, there are a few objectives we would like to achieve. Namely, we
would like to add better debugging and profiling functionality, expose
\gls{cuda} device initialization to the user, add automatic and/ or
configurable checkpointing, and finish node configuration and administrative
tasks.

\appendix
%% vim: syntax=tex:
\section{Appendix}

\subsection{Algorithms}

\begin{algorithm}
\begin{algorithmic}
\Function{Master}{$size,world\_size$}
    \Comment{Split data as appropriate and send to nodes}
    \State{$card\_max \gets query\_max\_mem()$}
    \State{$M \gets floor((N + card\_max - 1) / card\_max)$}
    \State{$m \gets floor((M + world\_size - 1) / world\_size)$}
    \ForAll{$r < world\_size$}
        \State{$slice\_low \gets (r * 2 + i) * card\_max$}
        \State{$slice\_high \gets (r * 2 + m) * card\_max$}
        \State{Send indices from low to high to node of rank $r$}
    \EndFor{}
\EndFunction{}
\Function{Minion}{$data,user\_fn$}
    \Comment{Receive Indices and Compute Results using user function}
    \State{$slice\_low \gets recv()$}
    \State{$slice\_high \gets recv()$}
    \State{$results \gets user\_fn(data[slice\_low:slice\_high])$}
    \State{Send back results or write them to disk}
\EndFunction{}
\end{algorithmic}
\label{alg:ocmf}
\caption{\texttt{OpenCUDA+MPI} Framework Pseudo Code}
\end{algorithm}

\begin{algorithm}
\begin{algorithmic}
\Function{main}{$rank$}
    \If{$rank == 0$} \State{$master(\dots{})$}
    \Else \State{$minion(\dots{})$}
    \EndIf{}
\EndFunction{}
\end{algorithmic}
\label{alg:run_code}
\caption{Combining the framework with user code}
\end{algorithm}

% vim: syntax=tex:
\section{Reference}
\renewcommand{\refname}{}
\bibliographystyle{plain}
\bibliography{references}

\end{document}
