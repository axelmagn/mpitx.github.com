% vim: syntax=tex:
\section{Methodology}

Our framework will be developed using a number of tools and will also build
upon a few already developed and mature software libraries. Overall, we plan to
use the \texttt{Python} programming language, \gls{mpi} for cross process
communication, and \gls{cuda} for graphics computing.

\subsection{Python}

We will use \texttt{Python} programming language because of its ease of
development and plethora of existing libraries such as \gls{mpi}.
\texttt{Python} also adds some advantages when it comes to usability when
needing more performance.  For example, if we discover a need for certain
aspects of the project to be tuned or otherwise run faster, we can easily
switch to C or C++ and write sections of the program in a (more performant)
native language.

\subsection{MPI}

Currently our cluster consists of 16 computers provided to our research lab by
the Computer Science department of Boise State University. We will use
\gls{mpi} because it has established itself as the de-facto interface for cross
process communication \cite{website:MPI-Tutorial}. To allow for interfacing
with \texttt{Python} \cite{website:mpi-4-python}, we will use \texttt{mpi4py}
because of its maturity and implementation completeness.

\subsection{CUDA}

Further, we will be using \gls{cuda} because of existing knowledge of the
framework and also because it is a well established framework for \gls{gpu}
computing.


\subsection{\texttt{CUDA+MPI}}

Our framework's goal is to be able to manage the job queue to be run on
\glspl{gpu} for a cluster of \gls{gpu}-capable computers. As more
implementation work is done one or more of the parameters listed below will be
considered to decide which particular job to run.

\begin{itemize}
    \item Job priority
    \item Compute resource availability
    \item Execution time allocated to user
    \item Number of simultaneous jobs allowed for a node
    \item Estimated execution time
    \item Elapsed execution time
    \item Availability of devices
\end{itemize}


\subsection{Testing}

To evaluate the efficacy and accuracy of our framework several tests will be
executed. As previously mentioned one of the first algorithms to be tested will
be the problem of vessel extraction. Accurately extracting vascular structures
from a Computerized Tomographic angiography--also called CT angiography
scans--is important for creating oncologic surgery planning tools as well as
medical visualization applications \cite{erdt2008automatic}.  Currently, we use
a single \gls{gpu} to extract vascular structures from a CT angiography scan,
which is computationally intensive. The following test programs will be
developed as time allows:

\begin{itemize}
    \item N-Body Simulation
    \item Prime Number Searching
\end{itemize}

Every test program will be evaluated in three categories: \gls{cpu}-only,
\gls{cuda}-only, and \texttt{CUDA+MPI}. Using the framework to accelerate the
computational tasks will provide us with key insights on its usability.
