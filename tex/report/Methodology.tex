% vim: syntax=tex:
\section{Methodology}

Our framework will be developed using a number of tools and will also build
upon a few already developed and mature software libraries. Overall, we plan to
use the \texttt{Python} programming language, \gls{mpi} for cross process
communication, and \gls{cuda} for graphics computing.

\subsection{Python}

We will use \texttt{Python} programming language because of its ease of
development and plethora of existing libraries such as \texttt{mpi4py} and
\texttt{pyCUDA}. \texttt{Python} also adds some advantages when it comes to
usability when needing more performance. For example, if we discover a need
for certain aspects of the project to be tuned or otherwise run faster, we can
easily switch to C or C++ and write sections of the program in a (more
performant) native language.

\subsection{MPI}

Currently our cluster consists of 16 computers provided to our research lab by
the Computer Science department of Boise State University. We will use
\gls{mpi} because it has established itself as the de-facto interface for cross
process communication \cite{website:MPI-Tutorial}. To allow for interfacing
with \texttt{Python} \cite{website:mpi-4-python}, we will use \texttt{mpi4py}
because of the library's maturity and implementation completeness.

\subsection{CUDA}

We will be using \gls{cuda}, and in particular, \texttt{pyCUDA}, because of
existing knowledge of the library/ framework and also because it is a well
established library for \gls{gpu} computing.

\subsection{\texttt{OpenCUDA+MPI}}

Our frameworks' goal is to make accessible the power of \gls{cluster} computing
without necessarily knowing in-depth the complexities of inter-computer related
communication. Further, in doing so, we would like to expose functionality that
may not be available even in more established \gls{cluster} libraries and
frameworks. Namely, facilities for debugging and profiling.

\subsection{Testing}

To evaluate the efficacy and accuracy of our framework several tests will be
executed. As previously mentioned one of the first algorithms to be tested will
be the problem of vessel extraction. Accurately extracting vascular structures
from a Computerized Tomographic angiography--also called CT angiography
scans--is important for creating oncologic surgery planning tools as well as
medical visualization applications \cite{erdt2008automatic}.  Currently, we use
a single \gls{gpu} to extract vascular structures from a CT angiography scan,
which is computationally intensive.

The following test programs will be developed as time allows:

\begin{itemize}
    \item N-Body Simulation
    \item Prime Number Searching
\end{itemize}

Every test program will be evaluated in three categories: \gls{cpu}-only,
\gls{cuda}-only, \gls{cuda} and \gls{mpi} and \texttt{OpenCUDA+MPI}. Having all
of the prior solutions that do \emph{not} use the framework provides a baseline
time comparison and provide immense insights into the pains and difficulties we
are actually attempting to solve.
