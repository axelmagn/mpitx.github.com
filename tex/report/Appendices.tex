% vim: syntax=tex:
\section{Appendix}

\subsection{Algorithms}

\begin{algorithm}
\begin{algorithmic}
\Function{Master}{$size,world\_size$}
    \Comment{Split data as appropriate and send to nodes}
    \State{$card\_max \gets query\_max\_mem()$}
    \State{$M \gets floor((N + card\_max - 1) / card\_max)$}
    \State{$m \gets floor((M + world\_size - 1) / world\_size)$}
    \ForAll{$r < world\_size$}
        \State{$slice\_low \gets (r * 2 + i) * card\_max$}
        \State{$slice\_high \gets (r * 2 + m) * card\_max$}
        \State{Send indices from low to high to node of rank $r$}
    \EndFor{}
\EndFunction{}
\Function{Minion}{$data,user\_fn$}
    \Comment{Receive Indices and Compute Results using user function}
    \State{$slice\_low \gets recv()$}
    \State{$slice\_high \gets recv()$}
    \State{$results \gets user\_fn(data[slice\_low:slice\_high])$}
    \State{Send back results or write them to disk}
\EndFunction{}
\end{algorithmic}
\caption{\texttt{OpenCUDA+MPI} Framework Pseudo Code}
\label{alg:ocmf}
\end{algorithm}

\begin{algorithm}
\begin{algorithmic}
\Function{main}{$rank$}
    \If{$rank == 0$} \State{$master(\dots{})$}
    \Else \State{$minion(\dots{})$}
    \EndIf{}
\EndFunction{}
\end{algorithmic}
\caption{Combining the framework with user code}
\label{alg:run_code}
\end{algorithm}
