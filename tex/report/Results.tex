% vim: syntax=tex:
\section{Results}

\subsection{Vector Summation}

Our first developed test program was a $ 10^9 $ element wise vector summation
problem. This is a simple and, as we will see, a bad example of using a cluster
to speed up the computational time required. Although we did see a increase in
performance, the cost of \gls{io} far out weighs the benefit. Specifically, to
do the computation on one machine (one \gls{cpu}), it took a \gls{wall_time} of
$254$ seconds (about $4$ minutes) and a \gls{cpu_time} of $13.7$ seconds.
Further, to do the computation using a single \gls{gpu} took about $4172$
seconds (\gls{wall_time}) or about $70$ minutes, $13.83$ seconds
(\gls{cpu_time}) while the computing the summation of the cluster took about
$3177$ seconds (\gls{wall_time}) or about $50$ minutes, $10.51$ seconds
(\gls{cpu_time}). Our \gls{cpu} only implementation took the least amount of
elapsed time, it took the second longest \gls{cpu_time}. Our \gls{cuda} only
implementation took the longest in both \gls{wall_time} and \gls{cpu_time} and
our cluster implementation was shortest \gls{cpu_time} but seconds longest
elapsed time. The benefit of saving an upwards of $3.3$ seconds is not worth
the extra incurred cost of $2923$ seconds. Not so surprisingly though, running
the vector summation problem over the cluster \emph{without} using \gls{cuda}
does increase the overall elapsed time of the problem; namely, it took $226$
seconds \gls{wall_time} (currently the correct \gls{cpu_time} is unable to be
collected). Further, increasing the number of nodes part of the program's pool,
decreases the \gls{wall_time} for each node.

\begin{table}[htb]
\centering{}
\begin{tabular}{lcc}
\toprule{}
\textbf{Method} & \textbf{Time (s)} & \textbf{Total Time (s)} \\
\midrule{}
CPU Only & 13.7 & 254.13 \\
\midrule{}
CUDA (Single \Gls{node}) & 13.83 & 4172 \\
\midrule{}
MPI + CUDA (7 \glspl{node}) & 10.51 & (average) 3177 \\
\midrule{}
MPI (7 \glspl{node}) & & (average) 226  \\
\midrule{}
MPI (16 \glspl{slot}) & & (average) 149 \\
\bottomrule{}
\end{tabular}
\caption{Computational Timing Comparison of $ 10^9 $ element wise vector
summation}
\end{table}

\subsection{N-Body Problem}

The \texttt{N-Body} problem is a simulation of a system of $N$ bodies and their
gravitational (or other force) interaction with each other calculated over a
set of time steps. Each time step will calculate and update positions and other
attributes of each body in the system.

We have several sizes of the \texttt{N-Body} problem that we tested with:
$2,001$ particles, $20,000$ particles, $200,000$ particles, $2,000,000$, and
$20,000,000$ particles.\\

The computational times are for a single time step. The method for computing
the gravitational potential is an adaptation of the \gls{p3m} method. The
benefit of using this method is we are able to nicely distribute the problem
over the \gls{cluster} and / or over a \gls{gpu} (because of memory
limitations) while maintaining a respectable accuracy for close body
interactions. Further, if a grid contains more bodies than a specified
threshold (in our case $200,000$), we can further sub-divide the grid to
improve performance and maintain accuracy still.\\

There are other algorithms for computing \texttt{N-Body} problems on the
\gls{cpu} that are quite efficient, notably, the Barnes-Hut Tree
algorithm\cite{barneshut1986}; however, using it would distort and confound the
comparisons between \gls{cpu}, \gls{gpu}, and \gls{cluster} implementations,
not to mention the complexities of implementing such an algorithm on
\glspl{gpu} and over a \gls{cluster}.

\subsubsection{N-Body --- CPU}

In \gls{cpu} tests, we were only able to complete several of the problem sizes;
the larger sizes are infeasible. Notably, the smaller sizes were computed in a
relatively respectable amount of time. While the bigger sizes were time
consuming, not even attempted, or aborted. For example, the 2 million body
problem was aborted after running for about 2 weeks.

\begin{table}[htb]
\centering{}
\begin{tabular}{lccc}
\toprule{}
\textbf{Size} & \textbf{User (seconds)} &
\textbf{Sys (seconds)} & \textbf{Real (seconds)} \\
\midrule{}
2001          & 28.81   & 0.02    & 30.77   \\
\midrule{}
20000         & 2382.40 & 2.27    & 2393    \\
\midrule{}
200000        & 113983  & 34.45   & 114349  \\
\midrule{}
2 million     & aborted & aborted & aborted \\
\midrule{}
20 million    & N/A     & N/A     & N/A     \\
\bottomrule{}
\end{tabular}
\caption{\gls{cpu} N-Body simulation using particle-particle method}
\label{tab:cpu_nbody}
\end{table}

\subsubsection{N-Body --- GPU}

As noted above, the \gls{gpu} (\gls{cuda}) implementation uses the same
computational method (\gls{p3m}). Using the \gls{gpu}, we were able to compute
the $ 20,000,000 $ size problem and may be able to compute larger sets within a
\emph{reasonable} amount of time. See \cref{tab:gpu_nbody} for a breakdown of
the running times.

\begin{table}[htb]
\centering{}
\begin{tabular}{lccc}
\toprule{}
\textbf{Size} & \textbf{User (seconds)} &
\textbf{Sys (seconds)} & \textbf{Real (seconds)} \\
\midrule{}
2001          & 10.08   & 1.52    & 14.14  \\
\midrule{}
20000         & 22.30   & 2.46    & 26.82  \\
\midrule{}
200000        & 44.63   & 4.84    & 52.86  \\
\midrule{}
2 million     & 186.59  & 21.23   & 217.60 \\
\midrule{}
20 million    & 1289.24 & 159.29  & 1510   \\
\bottomrule{}
\end{tabular}
\caption{Single \gls{gpu} (\gls{cuda}) N-Body simulation using \gls{p3m}
method}
\label{tab:gpu_nbody}
\end{table}

\subsubsection{N-Body --- 16 Node Cluster}

Similar to the other solutions, we are still using the \gls{p3m} method for
computing a single time step of the N-Body problem. Over 16 nodes, we were able
to see drastic improvements over \gls{cpu} and \gls{cuda} implementations.
Notably, in the $20,000$ problem size, the \gls{cluster} did nearly $19044\%$
better than the \gls{cpu} and about $115\%$ better than a single \gls{gpu}.
Further, in the $200,000$ problem size, we see the \gls{cluster} did about
$535,743\%$ better versus \gls{cpu} and about $148\%$ better versus \gls{gpu}.
See \cref{tab:cudampi_nbody} for times of each problem size.

\begin{table}[htb]
\centering{}
\begin{tabular}{lccc}
\toprule{}
\textbf{Size} & \textbf{User (seconds)} &
\textbf{Sys (seconds)} & \textbf{Real (seconds)} \\
\midrule{}
2001          & N/A     & N/A      & N/A     \\
\midrule{}
20000         & 0.17    & 0.032    & 12.50   \\
\midrule{}
200000        & 0.147   & 0.028    & 21.34   \\
\midrule{}
2 million     & 0.15    & 0.025    & 97.76   \\
\midrule{}
20 million    & 0.15    & 0.06     & 950.045 \\
\bottomrule{}
\end{tabular}
\caption{16 node \gls{cluster} N-Body simulation using \gls{p3m} method}
\label{tab:cudampi_nbody}
\end{table}
