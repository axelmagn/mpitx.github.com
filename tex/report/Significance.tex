% vim: syntax=tex:
\section{Significance}

\texttt{CUDA+MPI} will serve to better utilization of existing computing
resources. In other words, it will give us high compute power at low cost. Many
personal computers have a \gls{cuda} graphics card. It is more cost effective
to build a cluster of multiple computers than to purchase a supercomputer to
perform high performance computing. The reason for the low costs stem from the
fact that clusters are built of components which are sold in high volumes
\cite{erdt2008automatic}. \texttt{CUDA+MPI} will have the advantage of handling
heterogeneous distributed computing. In other words, \texttt{CUDA+MPI} avoids
hardware limitations. All the computers participating in the cluster do not
need to have the same model of \gls{cuda}-enabled \gls{gpu} cards. This will
also give us the freedom to easily add new nodes to the cluster as needed.

Since \texttt{CUDA+MPI} will operate in a heterogeneous cluster it will manage
the job requests from the user to be run on a heterogeneous cluster of
\gls{gpu}-capable computers. In order to decide job priority, amount of work
and availability of devices will be considered. In addition to improved
performance, the proposed framework will grant researchers and scientists the
assurance that the heterogeneous cluster will be optimized for computing
without requiring extra effort and thought.
